\documentclass[russian, 12pt]{article}
\usepackage[utf8x]{inputenc}
\usepackage[T2A]{fontenc}
\usepackage[russian]{babel}
\usepackage{amsfonts}
\usepackage{amssymb,amsmath,color}
\usepackage{wrapfig}
\usepackage{graphicx}
\usepackage{indentfirst}
\usepackage{commath}
\usepackage{multicol}
\usepackage{geometry}
\usepackage{textcomp}
\usepackage{esint}
\usepackage{mathrsfs}
\usepackage{float}
\usepackage{cancel}
\usepackage{empheq}
\usepackage{xr}
\usepackage{array}
\usepackage{tabu}
\usepackage{setspace}

\setcounter{page}{1}
\begin{document}

\section{Основные понятия}
\subsection{Событие.Вероятность события}
При рассмотрении опытов в которых могут быть различные исходы, результаты опытов будем называть событиями. Отличаем составные (разложимые) и элементарные (неразложимые) события.\\
\textbf{Пример} \\Выпало 6 очков при броске двух игральных костей -- составное разложение на (1, 5) или (2,4) или (3,3) или (4,2) или (5,1). \\
О том, что мы понимаем под элементарным событием надо предварительно условиться. Это неопределяемое понятие (как точка в геометрии). Они определяют идеализированный опыт. По определению, каждый  неразложимый исход идеализированного опыта представляется одним и только одним элементарным событием. Совокупность всеъ элементарных событий называется пространством элементарных событий  ($\sigma$). Элементарное событие -- точки этого пространства. Событие -- множество точек.\\
Совокупность точек представляет все те исходы, при которых происходит событие A, полностью описывают это событие. Любое множество точек А нашего пространства можно назвать событием; оно происходит или нет в зависимости от того, принадлежит или нет множеству A точка, представляемая исходного опыта.\\
\textbf{Пример}\\ Число курящих среди 100 человек. Пространство элементарных событий -- множество чисел 0, 1, 2,  . . . 100.\\
\textit{\textbf{Определение}}\\Невозможное событие -- это событие, которое в результате данного опыта не может произойти. Обозначим его $\emptyset$. Тоесть, запись A = $\emptyset$ означает, что А не содержит элементарных событий.\\
\textbf{Пример}\\Рассмотрим систему, состоящую из 6 атомов H. Выбираем один атом.\\
\textit{\textbf{Определение}} \\Достоверное событие -- то, которое в результате опыта  обязательно произойдет. Тоесть A = ($\sigma$).\\
\textbf{Пример}\\Достоверное событие -- выпадение $\leq$ 6 очков при броске одной игральной кости.\\
\textit{\textbf{Определение}}\\ Событие, состоящее из всех точек, не содержащих событие А, называется событием противоположным А и обозначается $\overline{A}$.\\
$\overline{\sigma}$ = $\emptyset$.\\
A -- выпадение орла, $\overline{A}$ -- решки.\\
\textit{\textbf{Определение}}\\ Суммой (объединением)  A + B $(A \cup B)$ событий А и B назовем событие, которое состоит в том, что имеет место или А, или В, или (и А, и В). Тоесть это обьединение множеств точек А и В.\\
%начало страницы 2
\textit{\textbf{Определение}}\\Произведением (пересечением) A $\cdot$ B $(A \cap B)$ событий A и B называется событие, которое состоит в том, что имеет место и А и В(одновременно) -- пересечение множества точек А и В.\\
\textit{\textbf{Определение}} \\События А и В несовместны если $(A \cup B)$ =  $\emptyset$.\\
(Тоесть не могут произойти одновременно)\\
\textbf{Пример\\}Бросок кости А -- не менее 3х очков, В -- не более 4х очков,$\overline{A}$ -- менее 3 очков(1 или 2), А + В = {$\sigma$}, A $\cdot$ B = 3 или 4 очка.\\
\textbf{Пример}\\Выстрел по мишени. А -- попадание, В -- промах,  A $\cdot$ B = $\emptyset$.\\
A + B = B + A\\
(A + B) + C = A + (B + C)\\
A $\cdot$ B = B  $\cdot$ A\\
\textit{\textbf{Определение}}\\Пространство элементарных событий называется дискретным, если оно состоит из конечного числа точек, или из бесконечного числа точек, которые могут быть занумерованы последовательно(Счетное число точек).\\
\textbf{Пример}\\Предыдущий пример -- конечное число точек.\\
Теперь попробуем ввести вероятность, то есть число, которое характеризует степень обьективной возможности события.\\
\textit{\textbf{Определение}}\\Пусть дано дискретное пространство элементарных событий {$\sigma$} с точками $E_{1}$, $E_{2}$, $E_{3}$ . . .  Предполагаем, что с каждой точкой $E_{i}$ (событием) связано число, называемое вероятностью $E_{i}$ и обозначаемое P($E_{i}$), такое что:\\ 
1)P($E_{i}$) $\geq$ 0  \\
2)P($E_{1}$) + P($E_{2}$) + . . . = 1\\
Вероятность любого события А есть сумма вероятностей элементарных событий из которых оно состоит.\\\\\\
1)P($\sigma$) = 1\\
2)P($\emptyset$) = 0\\
3)0 $\leq$ P(A) $\leq$ 1\\
%начало страницы 3
Как определить вероятность события в общей теории не постулируется. Об этом надо специально договариваться. Чаще всего встречается схема случаев.\\
Пусть  пространство эементарных событий состоит из n точек, причем все они равновозможны, тоесть  по условиям симметрии есть основание считать, что ни одно из них не является обьективно более возможным, чем другие. Напомним, кроме того, что элементарные  события  
несовместны. Такие элементарные события обычно называют случаями.\\
\textbf{Пример}\\Орел и решка при броске монеты. Появляется для любой  из карт тщательно перетасованной колоды.\\
Пусть событие А состоит из m точек (эти m случаев называются благоприятными событию A). Тогда вероятность P(A)  =$\frac{m}{n}$\\
\textbf{Пример}\\Бросок игральной кости. А  -- выпадение четного числа очков.\\n = 6, m = 3 (2, 4, 6) , следовательно, P(A) = $\frac{3}{6}$ = $\frac{1}{2}$ \\
В других сиуациях, не сводящихся к схеме случаях, вероятность определяется по другому(например плотник, землемер, штурман измеряют расстояния -- одно и тоже, но делают это по разному). При этом все способы 
с корнями уходят в опыт.\\
Пусть производится n опытов, в каждом из которых может появится событие А. Частотой события А называется отношение числа опытов, в которых появилось А к общему числу опытов. Частоту часто называют статистической вероятностью.\\ 
 0 $\leq$ $P^*$(A) $\leq$ 1, $P^*$(A) = $\frac{m}{n}$\\
Так определенная статистическая вероятность носит случайный характер. Но при росте n она стабильно около некоторого значения. При n  $\rightarrow$  $\infty$ с практической достоверностью (тоесть , вероятность ошибки сколь угодно мала) можно утверждать, что частота события будет сколько угодно мало отличаться от вероятности его в отдельном опыте. Более подробно это рассмотрим потом.\\
%начало страницы 4
\newpage
\begin{center}
$\textbf{Факты из комбинаторики}$
\end{center}
Число размещений с повторениями. $\overline{A^k_n}$ = $n^k$ , 
где n -- количество типов элементов -- группы по k элементов с учетом порядка(число трехзначных чисел в десятичной системе счисления равно $10^3$ - $10^2$)\\
Число размещений без повторений $A^k_n$ = $\frac{n!}{(n-k)!}$\\
$\textbf{Пример}$ \\12 человек учавствует в соревновании. Сколько вариантов распределения медалей. $\overline{A^3_{12}}$  = 12$\cdot$11$\cdot$10  = 1320\\
Число сочетаний без повторений  $C^k_n$ = $\frac{n!}{(n-k)!(k)!}$\\
$\textbf{Пример}$ \\12 команд. Сколько способов сформировать финальную группу из 3 команд без учета мест?\\
  $C^3_{12}$  = $\frac{ 12\cdot11\cdot 10 }{2\cdot3} $ = 220\\
Число  перестановок из n различных элементов $P_n$ = n! = ${A^n_n}$\\
Число перестановок из n = $n_1$ + $n_2$ +  $n_3$ . . .  + $n_k$ элементов, $n_1$ -- 1 типа, $n_2$ -- 2 типа, . . . $n_k$ -- k--го типа,\\
P($n_1$,  $n_2$, . . .  $n_k$) = $\frac{n!}{{n_1}!{n_2}!. . .{n_k}!}$\\
\textbf{Пример}\\ 8 ладей расстанавливаются на доске.Какова вероятность, что никакие две не бью друг -- друга?\\
Сколько способов расположить ладей на шахматной доске что бы они не били друг друга? На каждой горизонтальной по одной. Пусть  на первой горихонтали она стоит на позиции $a_1$,  на второй на позиции $a_2$ и так далее.\\
($a_1$, $a_2$, ... $a_8$)  -- перестановка чисел 1,2 . . . 8\\
Тоесть благоприятных случаев $P_8$ = 8!\\
P = $\frac{8!}{ 8^2 \cdot (8^2 - 1) \cdot (8^2 - 2)  \ .\ .\ . \   (8^2 - 7)} $  $\approx$   9 $\cdot$  $10^{-6}$
%начало страницы 5
Как определить вероятность если пространство элементарных собыий не является конечным? Чаcто здесь имеет смысл метод \underline{ геометрической вероятности}. Если пространство  $\sigma$ может быть изображено геометрической фигуры и по условию опыта вероятность попадания точки (элементарного события) в любую часть области $\sigma$ пропорционально мере этой части (длинне, площади, обьему . . .) и не зависит от ее расположения и формы, то вероятность события А определяется как P(A) = $\frac{S_A}{S}$, где $S_A$ -- мера части области, попадание в которую благоприятствует событию А, S - мера всей области.\\
$\textbf{Пример}$ Двое договорились встретится в определенном месте между 17 и 18 часами. Пришедший первым ждет второго 15 минут, после чего уходит. Определить вероятность встречи, если время прихода каждого независимо и равновероятно в течении этого часа\\
Благоприятные исходы : $\abs{x - y} \leq \frac{1}{4}$\\
$\frac{1}{4} \leq x - y \leq \frac{1}{4}$\\	
$S_{области}$ = $1 - {(1 - \frac{1}{4})}^2$\\
P = $\frac{1 - \frac{9}{16}}{1}$ = $\frac{7}{16}$
%начало страницы 6
\begin{center}
$\textbf{Парадокс де-Мере-Паскаля }$
\end{center}
Что вероятнее: при 3 бросках игральной кости получить в сумме: 11 или 12 очков?\\
Расуждение де-Мере: Cуммы 11 и 12 образуются при выпадении на костях следующих цифр: 12 =  6 +  5 + 1 = 6 + 3 + 3 = 5 + 4 + 3 = 5 + 5 + 2 = 4 + 4 + 4 (тоесть 6 вариантов);\\
11 =  6 + 4 + 1 = 6 + 3 + 2 = 5 + 5 + 1 = 5 + 4 + 2 = 5 + 3 + 3 = 4 + 4 + 2 (тоесть6 вариантов)\\
Тоесть 11 и 12 должны быть равновероятны, но на опыте 11 появляется чаще.\\
На ошибку указал Паскаль: необходимо учитывать все возможные комбинации цифр, дающие в сумме 11 или 12.





\end{document}
