\documentclass[russian, 12pt]{article}

\usepackage[utf8x]{inputenc}
\usepackage[T2A]{fontenc}
\usepackage[russian]{babel}
\usepackage{amsfonts}
\usepackage{amssymb,amsmath,color}
\usepackage{wrapfig}
\usepackage{graphicx}
\usepackage{indentfirst}
\usepackage{commath}
\usepackage{multicol}
\usepackage{geometry}
\usepackage{textcomp}
\usepackage{esint}
\usepackage{mathrsfs}
\usepackage{float}
\usepackage{cancel}
\usepackage{empheq}
\usepackage{xr}
\usepackage{array}
\usepackage{tabu}
\usepackage{setspace}


\DeclareGraphicsExtensions{.pdf,.png,.jpg}
\graphicspath{}
\setcounter{page}{1}
\begin{document}

\section{Основные понятия}
\subsection{Событие.Вероятность события}
При рассмотрении опытов в которых могут быть различные исходы, результаты опытов будем называть событиями. Отличаем составные (разложимые) и элементарные (неразложимые) события.\\
\textbf{Пример} \\Выпало 6 очков при броске двух игральных костей -- составное разложение на (1, 5) или (2,4) или (3,3) или (4,2) или (5,1). \\
О том, что мы понимаем под элементарным событием надо предварительно условиться. Это неопределяемое понятие (как точка в геометрии). Они определяют идеализированный опыт. По определению, каждый  неразложимый исход идеализированного опыта представляется одним и только одним элементарным событием. Совокупность всеъ элементарных событий называется пространством элементарных событий  ($\sigma$). Элементарное событие -- точки этого пространства. Событие -- множество точек.\\
Совокупность точек представляет все те исходы, при которых происходит событие $A$, полностью описывают это событие. Любое множество точек А нашего пространства можно назвать событием; оно происходит или нет в зависимости от того, принадлежит или нет множеству $A$ точка, представляющая исходный опыт.\\
\textbf{Пример}\\ Число курящих среди 100 человек. Пространство элементарных событий -- множество чисел 0, 1, 2,  . . . 100.\\
\textit{\textbf{Определение}}\\Невозможное событие -- это событие, которое в результате данного опыта не может произойти. Обозначим его $\emptyset$. То есть, запись $A$ = $\emptyset$ означает, что $A$ не содержит элементарных событий.\\
\textbf{Пример}\\Рассмотрим систему, состоящую из 6 атомов H. Выбираем один атом.\\
\textit{\textbf{Определение}} \\Достоверное событие -- то, которое в результате опыта  обязательно произойдет. То есть $A$ = ($\sigma$).\\
\textbf{Пример}\\Достоверное событие -- выпадение $\leq$ 6 очков при броске одной игральной кости.\\
\textit{\textbf{Определение}}\\ Событие, состоящее из всех точек, не содержащих событие А, называется событием противоположным А и обозначается $\overline{A}$.\\
$\overline{\sigma}$ = $\emptyset$.\\
$A$ -- выпадение орла, $\overline{A}$ -- решки.\\
\textit{\textbf{Определение}}\\ Суммой (объединением)  $A$ + $B$ $(A \cup B)$ событий $A$ и $B$ назовем событие, которое состоит в том, что имеет место или $A$, или $B$, или (и $A$, и $B$). То есть это обьединение множеств точек $A$ и $B$.\\
%начало страницы 2
\textit{\textbf{Определение}}\\Произведением (пересечением) $A$ $\cdot$ $B$ $(A \cap B)$ событий $A$ и $B$ называется событие, которое состоит в том, что имеет место и $A$ и $B$(одновременно) -- пересечение множества точек $A$ и $B$.\\
\includegraphics[scale=0.1]{x.png}\\
\textit{\textbf{Определение}} \\События $A$ и $B$ несовместны если $(A \cup B)$ =  $\emptyset$.\\
(То есть не могут произойти одновременно)\\
\textbf{Пример\\}Бросок кости $A$ -- не менее 3 очков, $В$ -- не более 4 очков,$\overline{A}$ -- менее 3 очков(1 или 2), $A$ + $B$ = {$\sigma$}, $A$ $\cdot$ $B$ = $3$ или $4$ очка.\\
\textbf{Пример}\\Выстрел по мишени. $A$ -- попадание, $B$ -- промах,  $A$ $\cdot$ $B$ = $\emptyset$.\\
$A$ + $B$ = $B$ + $A$\\
($A$ + $B$) + $C$ = $A$ + ($B$ + $C$)\\
$A$ $\cdot$ $B$ = $B$  $\cdot$ $A$\\
\textit{\textbf{Определение}}\\Пространство элементарных событий называется дискретным, если оно состоит из конечного числа точек, или из бесконечного числа точек, которые могут быть занумерованы последовательно(Счетное число точек).\\
\textbf{Пример}\\Предыдущий пример -- конечное число точек.\\
Теперь попробуем ввести вероятность, то есть число, которое характеризует степень обьективной возможности события.\\
\textit{\textbf{Определение}}\\Пусть дано дискретное пространство элементарных событий {$\sigma$} с точками $E_{1}$, $E_{2}$, $E_{3}$ . . .  Предполагаем, что с каждой точкой $E_{i}$ (событием) связано число, называемое вероятностью $E_{i}$ и обозначаемое P($E_{i}$), такое что:\\ 
1)$P$($E_{i}$) $\geq$ $0$  \\
2)$P$($E_{1}$) + $P$($E_{2}$) + . . . = $1$\\
Вероятность любого события $A$ есть сумма вероятностей элементарных событий из которых оно состоит.\\
1)$P(\sigma)$ = $1$\\
2)$P(\emptyset)$ = $0$\\
3)0 $\leq$ $P(A)$ $\leq$ $1$\\
%начало страницы 3
Как определить вероятность события в общей теории не постулируется. Об этом надо специально договариваться. Чаще всего встречается схема случаев.\\
Пусть  пространство эементарных событий состоит из n точек, причем все они равновозможны, то есть  по условиям симметрии есть основание считать, что ни одно из них не является обьективно более возможным, чем другие. Напомним, кроме того, что элементарные  события  
несовместны. Такие элементарные события обычно называют случаями.\\
\textbf{Пример}\\Орел и решка при броске монеты. Появляется для любой  из карт тщательно перетасованной колоды.\\
Пусть событие $А$ состоит из $m$ точек (эти $m$ случаев называются благоприятными событию $A$). Тогда вероятность $P(A)$  =$\frac{m}{n}$\\
\textbf{Пример}\\Бросок игральной кости. $A$  -- выпадение четного числа очков.\\$n$ =$6$, $m$ = $3$ ($2$, $4$, $6$) , следовательно, $P(A)$ = $\frac{3}{6}$ = $\frac{1}{2}$ \\
В других сиуациях, не сводящихся к схеме случаях, вероятность определяется по другому(например плотник, землемер, штурман измеряют расстояния -- одно и тоже, но делают это по разному). При этом все способы 
с корнями уходят в опыт.\\
Пусть производится n опытов, в каждом из которых может появится событие $А$. Частотой события $А$ называется отношение числа опытов, в которых появилось $А$ к общему числу опытов. Частоту часто называют статистической вероятностью.\\ 
 $0$ $\leq$ $P^*$($A$) $\leq$ 1, $P^*$($A$) = $\frac{m}{n}$\\
Так определенная статистическая вероятность носит случайный характер. Но при росте n она стабильно около некоторого значения. При n  $\rightarrow$  $\infty$ с практической достоверностью (то есть , вероятность ошибки сколь угодно мала) можно утверждать, что частота события будет сколько угодно мало отличаться от вероятности его в отдельном опыте. Более подробно это рассмотрим потом.\\
%начало страницы 4
\newpage
\begin{center}
$\textbf{Факты из комбинаторики}$
\end{center}
Число размещений с повторениями. $\overline{A^k_n}$ = $n^k$ , 
где $n$ -- количество типов элементов -- группы по $k$ элементов с учетом порядка(число трехзначных чисел в десятичной системе счисления равно $10^3$ - $10^2$).\\
Число размещений без повторений $A^k_n$ = $\frac{n!}{(n-k)!}$\\
$\textbf{Пример}$ \\$12$ человек учавствует в соревновании. Сколько вариантов распределения медалей. $\overline{A^3_{12}}$  = 12$\cdot$11$\cdot$10  = 1320\\
Число сочетаний без повторений  $C^k_n$ = $\frac{n!}{(n-k)!(k)!}$\\
$\textbf{Пример}$ \\$12$ команд. Сколько способов сформировать финальную группу из $3$ команд без учета мест?\\
  $C^3_{12}$  = $\frac{ 12\cdot11\cdot 10 }{2\cdot3} $ = $220$\\
Число  перестановок из $n$ различных элементов $P_n$ = n! = ${A^n_n}$\\
Число перестановок из $n$ = $n_1$ + $n_2$ +  $n_3$ . . .  + $n_k$ элементов, $n_1$ -- $1$ типа, $n_2$ -- $2$ типа, . . . $n_k$ -- $k$--го типа,\\
P($n_1$,  $n_2$, . . .  $n_k$) = $\frac{n!}{{n_1}!{n_2}!. . .{n_k}!}$\\
\textbf{Пример}\\ 8 ладей расстанавливаются на доске. Какова вероятность, что никакие две не бьют друг -- друга?\\
Сколько способов расположить ладей на шахматной доске что бы они не били друг друга? На каждой горизонтальной по одной. Пусть  на первой горихонтали она стоит на позиции $a_1$,  на второй на позиции $a_2$ и так далее.\\
($a_1$, $a_2$, ... $a_8$)  -- перестановка чисел $1,2\ .\ .\ .\ $8\\
То есть благоприятных случаев $P_8$ = $8$!\\
$P$ = $\frac{8!}{ 8^2 \cdot (8^2 - 1) \cdot (8^2 - 2)  \ .\ .\ . \   (8^2 - 7)} $  $\approx$   9 $\cdot$  $10^{-6}$\\
%начало страницы 5
Как определить вероятность если пространство элементарных собыий не является конечным? Чаcто здесь имеет смысл метод \underline{геометрической вероятности}. Если пространство  $\sigma$ может быть изображено геометрической фигуры и по условию опыта вероятность попадания точки (элементарного события) в любую часть области $\sigma$ пропорционально мере этой части (длинне, площади, обьему . . .) и не зависит от ее расположения и формы, то вероятность события А определяется как P(A) = $\frac{S_A}{S}$, где $S_A$ -- мера части области, попадание в которую благоприятствует событию $А$, $S$ -- мера всей области.\\
$\textbf{Пример}$\\Двое договорились встретится в определенном месте между 17 и 18 часами. Пришедший первым ждет второго 15 минут, после чего уходит. Определить вероятность встречи, если время прихода каждого независимо и равновероятно в течении этого часа.\\
\includegraphics[scale=0.2]{page5.png}\\
Благоприятные исходы : $\abs{x - y} \leq \frac{1}{4}$\\
$\frac{1}{4} \leq x - y \leq \frac{1}{4}$\\	
$S_{области}$ = $1 - {(1 - \frac{1}{4})}^2$\\
$P$ = $\frac{1 - \frac{9}{16}}{1}$ = $\frac{7}{16}$
%начало страницы 6
\begin{center}
$\textbf{Парадокс де-Мере-Паскаля }$
\end{center}
Что вероятнее: при $3$ бросках игральной кости получить в сумме: $11$ или $12$ очков?\\
Рассуждение де-Мере: Cуммы $11$ и $12$ образуются при выпадении на костях следующих цифр: $12$ $=$ $6$ $+$  $5$ $+$ $1$ $=$ $6$ $+$ $3$ $+$ $3$ $=$ $5$ $+$ $4$ $+$ $3$ $=$ $5$ $+$ $5$ $+$ $2$ $=$ \\
$4$ $+$ $4$ $+$ $4$ (то есть $6$ вариантов);\\
$11\ =\  6\ +\ 4\ +\ 1\ =\ 6\ +\ 3\ +\ 2\ =\ 5\ +\ 5\ +\ 1\ =\ 5\ +\ 4\ +\ 2\ =\ 5\ +\ 3\ +\ 3\ =\ 4\ +\ 4\ +\ 2\ $(то есть $6$ вариантов);\\
То есть $11$ и $12$ должны быть равновероятны, но на опыте 11 появляется чаще.\\
На ошибку указал Паскаль: необходимо учитывать все возможные комбинации цифр, дающие в сумме 11 или 12.\\
Например $6\ +\ 5\ +\ 1\ =\ 6\ +\ 1\ +\ 5\ =\ 5\ +\ 1\ +\ 6\ =\ 5\ +\  6\ +\ 1\ =\ 1\ +\ 5\ +\ 6$\\
$=\ 1\ +\ 6\ +\ 5$ (то есть $6$ способов = $3$!). Аналогично, $6\ +\ 4\ +\ 2\  $($6\ =\ 3!$ способов), $6\ +\ 3\ +\  3\ $($3$ способа), $5$ $+$ $4$ $+$ $3$ ($6$ способов), $4$ $+$ $4$ $+$ $4$ ($1$ способ).\\
$P_{11}$ = $\frac{6 +  6 + 6 + 3 + 3 + 3}{6^3}$  $\textgreater P_{12}$.\\
\begin{center}
$\textbf{Сравнение статистик Больцмана, Бозе -- Энштейна, Ферми -- Дирака }$
\end{center}
Дано $k$ частиц и $l$ ячеек ($l$ $\textgreater$ $k$)\\
Найти вероятность того что:\\
	1)в определенных $k$ ячейках окажется по 1 частице\\
	2)в каких то ячейках окажется по одной частице\\
\textbf{Cтатистика Больцмана}
\begin{tabbing}
Ей подчиняется обычный газ.\\
Условия:\\
\qquad a)частицы различны\\
\qquad б)в любой ячейке может находится сколько угодно частиц\\
\end{tabbing}
1)Общее число исходов $l^k$(Так как любую частицу можно положить в любую ячейку). Благоприятных исходов $k!$, так как частицы различны и их можно переставлять.\\
Значит $P_1$ = $\frac{k!}{l^k}$\\
2)Теперь можно $k$ ячеек выбирать из $l$. (То есть число сочетаний k из общего числа l).То есть число благоприятных исходов равно $C^k_l \cdot k!$\\
$P_2$ = $\frac{C_l^k\cdot k!}{l^k}$\\

\textbf{Cтатистика Бозе-Энштейна}
\begin{tabbing}
Ей подчиняется обычный газ.\\
Условия:\\
\qquad a)частицы различны\\
\qquad б)в любой ячейке может находится сколько угодно частиц\\
\end{tabbing}
%начало страницы 7
Общее число исходов.\\
Перестaвив ячейки в ряд, границы определим перегородками, которых $l + 1$. Если поменять местами две частицы, то нового распределения не получится, так как частицы неразличимы. Если поменять местами две перегородки, то тоже ничего нового не получится, так как все перегородки одинаковы. Если же поменять местами перегородку и частицу, то получится новое распределение. Две крайние перегородки закреплены, поэтому в перестановке учавствует $ l - 1$ перегородок и к частиц, то есть  $k + 1 - l$ элементов.\\
Число перестановок равно : $\frac{(k + l - 1)!}{k!\cdot (l - 1)!}$\\
Число благоприятных исходов равно:\\
1)Так как перестановка частиц не дает нового распределения, то благоприятный исход один(при фиксированных ячейках).\\
То есть $P_1$ =$\frac{1}{\frac{(k + l - 1)!}{k!\cdot(l-1)!}}$ = $\frac{k!\cdot(l-1)!}{(k + l - 1)!}$\\
2)Число благоприятных исходов равно числу способов выбрать $k$ ячеек из $l$, где будут частицы: $C^k_l$ = $\frac{l!}{k!(l - k)!}$\\
$P_2$ = $\frac{l!(k + l - 1)!}{k!(l - k)!k!(l - 1)!}$ \\
\\
\textbf{Cтатистика Ферми--Дирака}
\begin{tabbing}
Ей подчинен, нарпример, электронный газ.\\
\qquad a)частицы неразличимы\\
\qquad б)в ячейке может находится не более одной частицы (принцип Паули).
\end{tabbing}
Общее число исходов -- это число способов выбать из $l$ ячеек $k$, где будут частицы, то есть $C^k_l$ = $\frac{l!}{k!\cdot(l - k)!}$\\
Число благополучных исходов
\begin{tabbing}
\qquad 1)Так как $k$  ячеек определены, а частицы неразличимы, то благополучный исход один.\\$P_1$ = $\frac{1}{C^k_l}$ = $\frac{k!(l - k)!}{l!}$\\
\qquad 2)Число благополучных исходов равно числу способов выбрать $k$ заполненных ячеек из $l$ \\ равно $C^k_l$, следовательно $P_2 = \frac{C^k_l}{C^k_l}$ = 1
\end{tabbing}
%начало страницы 8
\subsection{Основные теоремы теории вероятностей}
$\textbf{Теорема сложения верояностей}$
\begin{tabbing}
\qquad\qquad$P(A + B) = P(A) + P(B)  -  P(AB)$
\end{tabbing}
\underline{Доказательство:}(для дискретного пространства элементарных событий). Что - бы получить $P(A + B)$ надо сложить вероятности точек входящих в $A$, и точек, входящих в $B$, на каждую по одному разу. Точки из $AB$ сосчитали дважды, поэтому их надо вычесть.\\
Если $A$ и $B$ несовместны то $P(A+B)=P(A)+P(B)$.
Если$A_1,\ A_2,\ .\ .\ .\ A_n$ - попарно несовместны, то $P(\sum\limits_{i=1}^{n}A_i)$ = $\sum\limits_{i=1}^{n}P(A_i)$\\
\textit{\textbf{Определение:}}\\ Говорят, что события $A_1, \ A_2,\ .\ .\ .\ A_n$ образуют полную группу, если $A_1+A_2+.\ .\ .\ A_n=\sigma$.\\
\textbf{Следстствие\ 1}\\ Если события $A_1,A_2,\ .\ .\ .\ A_n$ образуют полную группу попарно несовместных событий, то  $\sum\limits_{i=1}^{n}P(A_i)$ $=$ $1$\\
%???
\textbf{Следстствие\ 2} \\Сумма веротяностей противоположных событий равна $1$: $P(A)\ +\ P(\overline{A})\ =\ 1$\\
Если событий три:\\
$P(A+B+C)\  = \ P(A)\ +\ P(B)\ +\ P(C)\ -\ P(AB)\ -\  P(AC)\ -\ P(BC)\ +\ P(ABC)$\\
$P(\sum\limits_{i=1}^{n}A_i)$ = $\sum\limits_{i=1}^{n}P(A_i)$ - $\sum\limits{i,j}P(A_iA_j)$ $+$ . . . $+(-1)^{n-1}P(A_1,A_2, .\ .\ .\ A_n)$\\
$\textbf{Пример}$\\  Есть $100$ карточек с числами $1,\ 2,\ 3,\ .\ .\ .\ 100.$ Случайно выбирается одно из них.Событие $A$ -- число делится на $2$, $B$ -- делится на $3$. Найти вероятность $P(A+B)$.\\
\\
$P(A)\ =\ \frac{50}{100},\ P(B)\ =\ \frac{33}{100}$.\\  $AB$ -- делится на $6$: $P(AB)\ =\ \frac{16}{100}$. \\$P(A+B)\ =\ \frac{50}{100}\ + \frac{33}{100}\ -\frac{16}{100}=\frac{67}{100}$. \\\\
%начало страницы 9
\begin{center}
$\textbf{Торема умножения вероятностей }$
\end{center}
\textit{\textbf{Определение}}\\Событие $A$ называется независимым от события $B$, если вероятность события $A$ не зависит от того, произошло событие $B$ или нет, и зависимым, если вероятности $A$ меняются в зависимости от того, произошло событие $B$ или нет.\\
\textbf{Пример\ 1 }\\Бросание 2 монет. $A$ -- орел на первой монете, $B$ -- орел на второй монете. Эти события независимы.\\
\textbf{Пример\ 2 }\\Охотник, имеющий один патрон попадает в цель (cобытие $A$). Событие $B$ -- лев ловит добычу -- зависимые, если добыча -- охотник.\\
\textit{\textbf{Определение}}\\Вероятность события $A$, вычисленная при условии, что имело место событие $B$ -- называется условной вероятностью события $A$, и обозначается $P(A|B)$, или $P_B(A)$(считается $P(B)$ $\neq$ $A$). Условие независимости события $A$ от $B$ это $P(A|B)\ =\ P(B)$.\\
$\textbf{Теорема}$\\ $P(A\cdot B)\ =\ P(A)\cdot P(B|A)$.\\
\underline{Доказательство:}\\Для схемы случаев. Пусть в результате опыта возможно $n$ исходов, событие $A$ благоприятных $m$ исходов, событие $B$ -- $k$ исходов, и событие $A$, и событие $B$ -- $l$ исходов. $P(A\cdot B)\ =\ \frac{l}{n}$, $P(A)\ =\ \frac{m}{n}$. Если известно, что $A$ происходит, то из $n$ остальных возможных исходов, из них $l$ блокируют событие $B$, то есть  $P(A|B)=\frac{l}{m}$. То есть $P(A\cdot B) = P(A)\cdot P(B)$.\\
Ясно, что $P(A\cdot B) = P(B)\cdot P(A|B)$\\
\textbf{Следстствие\ 1}\\Если событие $A$ не зависит от события $B$, то и событие $B$ не зависит от события $A$.
\begin{tabbing}
Известно, что $P(A)$ = $P(A|B)$. Считаем, что $P(A)$ $\neq$ $0$.\\
\end{tabbing}
\begin{equation*} 
 \begin{cases}
   $P(AB)$ = $P(A)P(B|A)$\\
   $P(AB)$ = $P(B)P(A|B)$
 \end{cases}
\end{equation*}
Следовательно, $P(A)P(B|A)$ = $P(B)P(A)$, следовательно\\
(при условии $P(A)$ $\neq$ $0$) $P(B|A)=P(B)$\\
То есть, свойство зависимости или независимости взаимно.\\
\textbf{Следстствие\ 2}\\ Для независимых событий $P(A\cdot B)$ $=$ $P(A)\cdot P(B)$\\
Замечание: Эта формула может быть взята за определение независимых событий.\\
Умножение для $n$ событий.\\ $P(A_1, A_2, .\ .\ .\ .\ , A_n)$=$P(A_1)\cdot P(A_2|A_1) \cdot P(A_3|A_1A_2) \ .\ .\ . \ P(A_n|A_1,A_2 .\ .\ A_n)$\\
\textit{\textbf{Определение}}\\События $A_1, A_2, .\ .\ .\ , A_n$ называются независимыми в совокупности, если :\\ $P(A_1)\cdot P(A_2) \cdot .\ .\ .\  P(A_n)$\\
Замечание: это определение эквивалентно следующему: события независимы в совокупности, если любое из них не зависит любой совокупности остальных.\\
%начало страницы 10
$\underline{Замечание}$ Независимость в совокупности не эквивалентна попарной независимости событий\\
\textbf{Пример\ 1 }(Пример Берштейна): \\Имеется 4 шара. Красного, желтого, зеленого и трехцветный, имеющий красный, жёлтый и зелёный цвета на себе.Событие К -  вынули шар, на котором красный цвет, событие Ж - есть желтый, событие З - зелёный. $P($K$)=\frac{1}{2}$ = $P($Ж$)$ = $P($З$)$\\
Вероятность того, что вынули шар одновременно с двумя цветами = $P($K$\cdot$Ж$)\ =\frac{1}{4}$ = $P($К$)\cdot P($Ж$)$,  $P($K$\cdot$З$)\ =$  $P($К$)\cdot P($З$)\  = \  \frac{1}{4}$, $P($Ж$\cdot$З$)\ =$  $P($Ж$)\cdot P($З$)\  = \  \frac{1}{4}$.Вероятность того, что выпадет три цвета равна $P($К$\cdot$Ж$\cdot$З$)$ = $\frac{1}{4}\neq\\ \neq\ P($К$)\cdot P($Ж$) \cdot P($З$)=\frac{1}{8}$ То есть они попарно независимы, но зависимы в совокупности.\\
\textbf{Пример\ 2}\\
Техническое устройство отказывает с вероятностью $p=0.5$. Сколько раз его надо продублировать, что бы вероятность отказа установки была $q < 0.1$?\\
$P=p^n<0.1=q$, $n > \frac{ln(q)}{ln(p)}$\\
Суммы и произведение вероятностней часто работают вместе.\\
\textbf{Пример\ 3}\\ Пусть все элементы отказывают независимо от комбинаций других.Чему равна вероятность отказа цепи?\\
$P(A+B_1B_2+C_1C_2C_3)=P(A)+P(B_1)\cdot P(B_2)+P(C_1C_2C_3)-P(AB_1B_2)-P(AC_1C_2C_3)-P(B_1B_2C_1C_2C_3)+P(AB_1B_2C_1C_2C_3)$\\
\newpage
\begin{center}
$\textbf{Формула полной вероятности }$
\end{center}
\textit{\textbf{Определение}}\\Говорят, что события $H_1$, $H_2$, . . . $H_n$ образуют полную группу, если $H_1+H_2+$. . .$H_n$ -- достоверное событие.\\
Пусть $H_1$, $H_2$, . . . $H_n$ -- полная группа попарно несовместимых событий. Эти события будем называть гипотезами. Пусть надо найти вероятность события $A$, которое может произойти вместе с одной из гипотез. Тогда $P(A)=\sum\limits_{i=1}^{n}P(H_i)\cdot P(A|H_i)$\\
Так как $H_1$, . . .$H_n$ -- полная группа. $A$ $=$ $H_1A\ +H_2A\ +$ . . . $+H_nA$.$H_1,$. . .$H_n$ -- попарно несовместные, следовательно $H_1A,$ . . . $H_nA$ -- несовместны, следовательно $P(A)=P(H_1A) +$. . . $P(H_nA)$ $=$ $\sum\limits_{i = 1}^{n} P(H_iA)$, следовательно, по теореме умножения $P(A)=\sum\limits_{i = 1}^{n}P(H_i)\cdot P(A|H_i)$\\
\textbf{Пример\ }\\По самолету производится 3 выстрела. Вероятность попадания при первом $0.4$, при втором - $0.5$, при третьем -- $0.7$. Для вывода самолета из
%начало страницы 11
 строя завведомо достаточно 3 попаданий. При первом попадании самолет выходит из строя с вероятностью $0.2$, при двух -- $0.6$. Найти вероятность того, что в результате трех выстрелов самолет будет выведен из строя.\\
$H_i$ -- в самолет попал $i$--й снаряд, $P(H_0)=0.6\cdot0.5\cdot0.3=0.09$. $P(H_1)=0.4\cdot0.5\cdot0.3+0.6\cdot0.5\cdot0.3+0.6\cdot0.5\cdot0.7=0.36$\\
$P(H_2)=0.6\cdot0.5\cdot\cdot0.7+0.4\cdot0.5\cdot0.7+0.4\cdot0.5\cdot0.5=0.41$, $P(H_3)=0.4\cdot0.5\cdot0.7=0.14$\\
$P(A)=0.36\cdot0.2+0.41\cdot0.6+0.14\cdot1=0.458$\\
\begin{center}
$\textbf{Формула Байеса }$
\end{center}
Пусть событие $A$ может произойти с одним из $n$ попарно несовместимых событий $H_1, H_2, .\ .\ .\ H_n$, образующих полную группу. Вероятности гипотез до опыта известны -- $P(H_1)$, . . . $P(H_n)$ (априорные вероятности). Произведен опыт, в результате которого произошло событие $A$. Как следует изменить вероятности гипотез в связи с появлением $A$ (то есть найти постериорные вероятности $P(H_i|A)$?
$P(AH_i)=P(A)\cdot P(H_i|A) = P(H_i) \cdot P(A|H_i)$, следовательно $P(H_i|A) = \frac{P(H_i) \cdot P(A|H_i)}{P(A)}$.\\
Итого: $P(H_i|A) = \frac{P(H_i)\cdot P(A|H_i)}{ \sum\limits_{j=1}^{n}P(H_j)\cdot P(A|H_j) }$\\
\\
\textbf{Пример\ }\\
В первой урне 5 белых и 10 черных шаров.Во второй - 3 белых и 7 черных. Из второй в первую переложили $1$ шар, а затем из второй вынули один шар. Оказалось что он белый. Найти вероятность того, что был переложен белый шар\\
$H_1$ -- переложили белый, $H_2$ -- черный, $P(H_1)=\frac{3}{10}$, $P(H_2)=\frac{7}{10}$. $P(A|H_1)=\frac{6}{11}$, $P(A|H_2)=\frac{5}{11}$\\
 $P(A)=\frac{3}{10}\cdot \frac{6}{11} + \frac{7}{10} \cdot \frac{5}{11}$, $P(H_1|A)$ = $\frac {\frac{3}{10}\cdot \frac{6}{11} }   {\frac{3\cdot 6}{10\cdot11} + \frac{7\cdot5}{10\cdot11}  }$ = $\frac{18}{18 + 35}$\\
\textbf{Пример\ }\\
Известно, что $5\%$ мужчин и $0.25\%$ женщин -- дальтоники. Наугад выбранное лицо страдает дальтонизмом. Какова вероятность того, что это мужчина?\\
$P(H_1) = P(H_2) = \frac{1}{2}$\\
$P(A|H_1) = 0.05$, $P(A|H_2) = 0.0025$, $P(H_1|A) = \frac{\frac{1}{2} \cdot 0.05}{ \frac{1}{2} \cdot 0.05 + \frac{1}{2} \cdot 0.0025 } = \frac{20}{21}$\\
%начало страницы 12
\subsection{Повторение опытов}
В самом начале развития теории вероятностей выяснилась фундаментальная роль одной математической схемы, изученной швейцарцем Яковом Бернулли.\\
Схема такая: проведем последовательность испытаний, в каждом из которых вероятность события $A$ одна и та -- же($p$). Испытания независимы, то есть вероятность воявления события $A$ в каждом из них не зависит от того, появилось оно или нет в других испытаниях.\\
\textbf{Пример\ }\\ $2$ игрока играют в шахматы $3$ партии. Вероятность выигрыша первого $p = \frac{2}{3}$. Найти вероятность того, что он выиграет $2$ партии.\\
Это можно осуществить : $P(A_1A_2\overline{A_3} + A_1\overline{A_2}A_3 + \overline{A_1}A_2A_3)=\frac{2}{3}\cdot\frac{2}{3}\cdot\frac{1}{3} + \frac{2}{3}\cdot\frac{1}{3}\cdot\frac{2}{3} + \frac{1}{3}\cdot\frac{2}{3}\cdot\frac{2}{3} = \frac{4}{9}.$\\
В общем виде: вероятность того, что в $n$ опытах событие произойдет $m$ раз $P_n(m)$ равна $A_1A_2.\ . \ . \ A_m\overline{A_{m+1}}.\ .\ .\ \overline{A_n} + 
A_1A_2.\ . \ . \ \overline{A_m} A_{m+1} \overline{A_{m+2}} .\ .\ .\ \overline{A_n} +
\ \ \ .\ .\ .\ 
+\overline{A_1}\overline{A_2}.\ . \ . \ \overline{A_{n-m}} {A_{n-m+1}} .\ .\ .\ A_n$
В каждую комбинацию $A$ входит $m$ раз %****
Число комбинаций $C^m_n$, все комбинации несовместны, следовательно $P_n(m) = p^mq^{n-m} + p^mq^{n-m} + .\ .\ . = C^m_np^mq^{n-m}$ , где $q = 1 - p$.\\
Так как по форме $P_n(m)$ %не понял
член разложения бинома $(q+p)^n$ распределение веротяностей такого вида называется биноминальным распределением.\\
\textbf{Пример\ }\\
Два шахматных игрока, 10 результативных партий (ничьи не учитываются), Вероятность выигрыша первого -- $\frac{2}{3}$, второго -- $\frac{1}{3}$. Найти вероятность выигрыша
 всей игры первым?\\
$P_{выигрыша\ 1} = P_{10}(6) + P_{10}(7) + P_{10}(8) + P_{10}(9) +  P_{10}(10)$ = $\frac{2^6}{3^{10}}\cdot(210+240+180+80+16) = \frac{2^6\cdot241}{3^{9}}$
$P_{выигрыша\ 2} = P_{10}(6) + P_{10}(0) + P_{10}(1) + P_{10}(2) +  P_{10}(3)  +  P_{10}(4)$ = $\frac{1507}{3^{9}}\cdot(210+240+180+80+16)$\\
То есть вероятность выигрыша первой партии у первого в два раза больше чем у второго, вероятность выигрыша матча у первого в 10 раз больше, чем у второго.\\
\textbf{Замечание\ } \\
$\sum\limits_{m = 0}^{n}P_n(m) = 1$, $(p+q)^n=1$, $p+q = 1$, следовательно, $(p+q)^n = \sum\limits_{m = 0}^{n}P_n(m)  = \sum\limits_{m = 0}^{n}C^m_np^mq^{(n-m)}$ -- Бином Ньютона.\\
%начало страницы 13
\section{Случайные величины}
\textit{\textbf{Определение}}\\
Функция, определённая на пространстве элементарных событий называется случайной величиной.\\
\textit{\textbf{Определение}}\\
Случайная величина называется дискретной, если она определена на дикретном пространстве элементарных событий.\\
Замечание: лучше бы было называть случайные величины функциями случая\\
\textbf{Пример\ }\\Дискретная случайная величина. Число тузов у одного игрока при игре в бридж, чтсло совпадающих дней рождения в группе из $n$ человек.\\
Обозначать случайные величины будем буквами $X, Y$, . . . и их значения $x, y$ . . . .
Пусть $X$ -- случайная величина. $x_1, x_2,$ . . .  --  ее значения. Совокупность всех элементарных событий, на которых $X$ принимает значение $x_i$ образует событие $X=x_i$.
Его вероятность обозначается $P(X=x_j) = p_j$. Соотношение, устанавливающее связь между значениями случайных величин и их вероятностями называется законом распределения случайной величины. Самой простой формой закона распределения для случайной величины является ряд распределения то есть таблица распределения, в которой сведены значения случайной величины и их вероятности. Для наглядности это часто изображают на графике и точки соединяют отрезками прямых. Получившаяся фигура называется многоугольником распределения. Так как события $X=x_j$ -- несовместимы и образуют полную группу, то $\sum\limits_{i = 1}^{n}p_i = 1$.\\
\textbf{Пример\ }\\
Баскетболист бросает мяч в кольцо до первого попадания, либо пока не сделано 3 броска. Вероятность попадания при одном броске равна $0.7$.\\
%началоо страницы 14

\end{document}
